Type I X-ray bursts (XRBs) are the thermonuclear runaway of a thin
layer of hydrogen and/or helium on the surface of a neutron star. 
This fuel layer accretes from a binary companion star and the immense
gravitational acceleration on the surface of the neutron star
compresses it, increasing the temperature and density to the point of
explosion.  One-dimensional hydrodynamic studies have been able to
reproduce many of the observable features of XRBs such as burst
energies ($\sim 10^{39}$ erg), rise times (seconds), durations ($10$'s
-- $100$'s of seconds) and recurrence times (hours to days) (see
\cite{STRO_BILD06} for an overview of XRBs).  By construction,
however, 1D models assume that the fuel is burned
uniformly over the surface of the star which is unlikely if the
accretion is not spherically symmetric \cite{SHARA82}.  Furthermore,
the {\em Rossi X-ray Timing Explorer} satellite has observed coherent
oscillations in the lightcurves of $\gtrsim 20$ outbursts from low-mass
X-ray binary
systems (first by \cite{STRO_ETAL96}; more recently by
\cite{ALTAMIRANO_ETAL10} and references therein).  The asymptotic
evolution of the frequency of such oscillations suggests they are
modulated by the neutron star spin frequency \cite{MUNO_ETAL02} and
are therefore indicative of a spreading burning front being brought in
and out of view by stellar rotation.

Before the actual outburst, the burning at the base of the ignition
column will drive convection throughout the overlying layers and set
the state of the material in which the burning front will propagate.
One-dimensional simulations of XRBs usually attempt to parametrize the
convective overturn and mixing using astrophysical mixing-length
theory or through various diffusive processes
(e.g. \cite{HEGER_ETAL00}).  A proper treatment of the convection in
these extreme conditions, free from parameterizations, requires 3D
simulations.  We completed the most detailed (2D) studies of
convection in pure He bursts~\cite{XRB-paper} and mixed H/He
bursts~\cite{XRB2}, and have just begun 3D simulations.  3D is
expensive, restricting us to small domains to resolve the burning
processes and small reaction networks (10-isotopes currently) to fit
the model in memory.  In small domains, the convection ``senses'' 
the boundaries, which can alter the dynamics.  Ultimately we want
to model a wide domain with many convective plumes side-by-side
and a more realistic reaction network---this is what we will
accomplish with our proposed simulations.

