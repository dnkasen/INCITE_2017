\documentclass[11pt,letterpaper,english]{article}
\usepackage[T1]{fontenc} % Standard package for selecting font encodings
\usepackage{txfonts} % makes spacing between characters space correctly
\usepackage{xcolor} % Driver-independent color extensions for LaTeX and pdfLaTeX.
\usepackage{hyperref}  %The ability to create hyperlinks within the document

\usepackage{fancyhdr} % header footer placement

\usepackage[top=1in, bottom=1in, left=1in, right=1in] {geometry} % Margins
\usepackage{graphicx}   % Essential for adding images to you document.

\usepackage[numbers]{natbib}

%\usepackage{sectsty}
%% \sectionfont{\normalsize}
%% \subsectionfont{\normalsize}
%\subsubsectionfont{\normalsize \it}

\usepackage[small,compact]{titlesec}
\titlespacing{\subsubsection}{0pt}{0.5em}{0.25em}
\titlespacing{\subsection}{0pt}{0.5em}{0.5em}
\titlespacing{\section}{0pt}{0.5em}{1.0em}
\titlespacing{\paragraph}{0pt}{0.75em}{0.25em}

\titleformat*{\subsubsection}{\itshape}


\let\oldthebibliography=\thebibliography
  \let\endoldthebibliography=\endthebibliography
  \renewenvironment{thebibliography}[1]{%
    \begin{oldthebibliography}{#1}%
      \setlength{\parskip}{0.25ex}% 
      \setlength{\itemsep}{0.25ex}% 
  }%
  {% 
    \end{oldthebibliography}% 
  }



\usepackage{caption}
\captionsetup{labelsep=period}

\setlength{\parskip}{0.125\baselineskip}%
%\setlength{\parindent}{0pt}%

\input newcommands

%\raggedright


\begin{document}

%\setlength{\parindent}{0in} % Amount of indentation at the first line of a paragraph.

\pagestyle{fancy} 
\lhead{Approaching Exascale Models of Astrophysical Explosions} 
\rhead{PI: Zingale} \renewcommand{%
\headrulewidth}{0.0pt}


\begin{center}
\bf {DATA MANAGEMENT}
\end{center}


%% \item Describe the data your research uses and generates.\\


%% \item What is the effective lifetime of your useful data? Do your
%% data storage requirements entail preserving data for weeks? Months?
%% Years? \\

Each simulation can output 100s of TB of data.  We need the ability to
process all of this data at the end of a simulation, so we store the
plotfiles to HPSS as they are created.  A single scientific study
involves several simulations, and factoring in queue times on the
machines, it can take a year to complete a study.  At that point we
finish the analysis and write up a paper describing the results.  Once
we get done with the refereeing process, it can be 1.5 years since a
simulation began.  This defines the minimum timescale over which we
need access to the data.

%% \item What types of tools for data storage, compression
%% (reduction), movement, and analysis do you currently use?\\

To reduce our data footprint, we can store the plotfile data in single
precision, and we keep only a minimum set of checkpoint files
archived.  All of the analysis is done remotely, as the dataset is to
large to transfer.  In addition to homegrown analysis tools and basic
runtime diagnostics (global quantities at each timestep), we use
\visit\ and, increasingly, \yt\ for our visualization.  This methodology
has served well during our current INCITE allocation.

There is another important aspect to our data management.  All of our
codes are managed by version control (git) and we use regular
regression testing to ensure that the code continues to give the same
answers as it is developed.  Furthermore, our codes are publicly
available and our output files store all the information needed to
reproduce a simulation (the git hashes for all code, the build machine
and directory, the build and run dates, number of processors and
threads, the compiler versions and flags, and the values of all
runtime parameters).  This gives us strong reproducibility of our
results---in the event of data loss or if we wish to revisit a
problem, all the information needed to reproduce a result is at hand.

%% \item Do you share data? If yes, what mechanisms are employed? \\ 

We have not encountered any requests for data from a simulation, but we
do make available all of the setup files needed for the various simulations
once the primary science papers are published (typically this is to allow
for the graduate student whose project it is to complete their thesis).
For example, al the setup, inputs, microphysics, and model files for the Chandra
WD convection problem that was a focus of our current INCITE are part of the
standard \maestro\ code distribution.

%% \item What types of data management services would be useful to
%% further enable your research?\\



%% The Leadership Computing centers are combining their existing data
%% efforts into an integrated LCF data strategy to meet the needs of the
%% growing class of large data science problems for which the volume and
%% velocity of the data require the resources available only at the
%% LCF. As part of the discovery and requirements-gathering activity,
%% your input will help steer the direction of resource acquisition,
%% tools development, and policy decisions.

%% A data management plan (DMP) describes how a project will manage the
%% data generated through the course of research. It should address how
%% data will be shared and preserved or explain why these goals are not
%% applicable.

%% The INCITE program plans to implement a requirement for a formal DMP
%% as part of future proposals. A DMP is not currently required; however,
%% for the 2015 INCITE Call for Proposals, please submit a short document
%% (not to exceed 1 page) that describes your expected future data
%% management strategies and needs.  Your summary should include comments
%% on the following topics:


%% \vspace{-.15in}
%% \begin{itemize}






%% \end{itemize} 
%% \vspace{-.15in}


\end{document}
