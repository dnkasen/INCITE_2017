Type Ia supernovae (SNe Ia) are the thermonuclear explosion of a  \MarginPar{needs updating, shortening--- How's this?}
carbon/oxygen white dwarf (WD) in a binary system.  These 
bright explosions rival the luminosity of the host galaxy,
making them visible at cosmological distances in the Universe. Furthermore,
the brightest events take the longest to dim, which allows their 
use as distance indicators and led to discovering
the acceleration of the expansion of the Universe, a
remarkable advance.

A fundamental uncertainty in our understanding of SNe Ia is the nature
of the progenitor---a single white dwarf accreting from a normal
companion star (single degenerate scenario) or two white dwarfs that
inspiral (or violently collide) and merge (the double degenerate
scenario) (for a review of explosion models, see \cite(calder2013}).
No conclusive of the progenitor system has been made,
but in every case the majority of the carbon and
oxygen in the white dwarf(s) is converted into iron/nickel and
intermediate-mass elements like silicon, and this nuclear energy
release unbinds the star.

For many years, the near-Chandrasekhar-mass single degenerate model
(henceforth called the Chandra model) saw the most attention.  The
Chandrasekhar mass is the maximum possible mass of a white
dwarf, so explosions near this limit imply SNe Ia would be alike. 
Contemporary observations suggest diversity and it is not 
clear if nature makes SNe Ia this way---massive
carbon/oxygen white dwarfs in binary systems are rare.  Accordingly,
contemporary research explores other explosion models. Our proposal
focuses on three aspects of this problem: merging white dwarfs
and sub-Chandra explosion models and the convective Urca process
thought to occur during the late stage of accretion in the 
Chandra model. These three problems are in relatively early stages
of research exploration, and chances to make fundamental 
contributions abound. 

The standard picture for double degenerate SN Ia (WDWD) has the
white dwarfs inspiraling as gravitational radiation
removes orbital energy.  The less massive star will
become tidally disrupted and the more massive star will accrete this
material.  A longstanding concern is that when this mass transfer
begins, thermonuclear burning can ignite at the edge of the star,
converting it to oxygen/neon/magnesium, and leading to the collapse of
the white dwarf into a neutron
star~\cite{saionomoto:2004,fryerdiehl:2008}, instead of an SNe Ia.
This system is inherently three-dimensional, and only through detailed
simulation can we understand the dynamics of the mass transfer, and
thereby assess the feasibility of this model.  

The other alternative mechanism, sub-Chandra mass single degenerate SNe
Ia~\cite{fink:2010,shen:2010,sim:2012} (sub-Ch
model), has the advantage that systems with a moderate mass white dwarf
(0.8--1.0 solar masses) are known and abundant, and this model also can
reproduce the known delay-time distributions and SNe Ia
rates---something the Chandra model struggles with~\cite{ruiter:2011}.
In this model, the white dwarf accretes a layer
of helium on its surface that detonates, sending 
a compression wave into the
underlying carbon/oxygen white dwarf that subsequently 
ignites a detonation in the core that unbinds the star. The main
problem with this model is that it typically produces abundances
and spectra that do not match observations 
well~\cite{hoeflich:1996,nugent:1997,kromer:2010}.  
It has been suggested
that a detonation in a very thin He layer can trigger the detonation
of the core without over-producing surface iron-group
elements~\cite{fink:2010}, and the question is understanding under 
what conditions ignition can take
place in these thin shells when a realistic 3D convective field is
realized, as well as the subsequent evolution and character of that
ignition---this is what we will answer.

The final piece of the Ia puzzle we will explore is the 
the earliest stage of the convection in the white
dwarf as it approaches the Chandrasekhar limit in the Chandra
model, when neutrino losses in the reaction can alter the dynamics of
the convection (called the URCA process).  The only multi-dimensional
simulations of URCA~\cite{URCA} to date have been 2D, with only a
portion of the star modeled.  Applying the low-Mach code MAESTRO
to this problem will allow simulations with unprecedented realism.
