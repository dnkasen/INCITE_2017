Type Ia supernovae (SNe Ia) are the thermonuclear explosion of a  \MarginPar{needs updating, shortening}
carbon/oxygen white dwarf (WD) in a binary system.  These are incredibly
bright explosions that rival the light output of their host galaxy,
making them visible at vast distances in the Universe.  Furthermore,
they have an interesting property: the brightest events take the
longest to dim.  This allows them to be used as distance indicators
and led to the discovery of the acceleration of the expansion of the
Universe. 

A fundamental uncertainty in our understanding of SNe Ia is the nature
of the progenitor---a single white dwarf accreting from a normal
companion star (single degenerate scenario) or two white dwarfs that
inspiral (or violently collide) and merge (the double degenerate
scenario).  No progenitor system of an SN Ia has ever been
identified, so astronomers must look for indirect clues.  Regardless
of the progenitor system, in every case the majority of the carbon and
oxygen in the white dwarf(s) is converted into iron/nickel and
intermediate-mass elements like silicon, and this nuclear energy
release unbinds the star.

For many years, the Chandrasekhar-mass single degenerate model
(henceforth called the Chandra model) saw the most attention.  The
Chandrasekhar mass is the maximum possible mass of a white
dwarf---beyond this mass, the degenerate electrons that provide the
pressure support against gravity in a white dwarf succumb to the
weight of the star and it will collapse into an even more dense
object: a neutron star.  By always exploding near the Chandrasekhar
mass, the same amount of fuel will be involved in every event and to
first-order all SNe Ia would be alike.  Many groups
(e.g.~\cite{gamezo:2005,Roe07,Jor08}, including us~\cite{Kru12,Ma13})
have modeled the explosion (and also the convective stage
preceding it~\cite{hoflichstein:2002}; again including us~\cite{Non12}) and have shown that it can reproduce the
spectra and lightcurves of the observations~(e.g.~\cite{Blo11}).
However, it is not clear if nature makes SNe Ia this way---massive
carbon/oxygen white dwarfs in binary systems are rare.  An additional
complication is that the successful models rely on a
deflagration-to-detonation transition (subsonic to supersonic
burning), the physics of which is not completely understood.  Buoyed by
observations that showed SNe Ia are more diverse than previously
thought, and evidence that there may be two
populations~\cite{MannucciEtAl06,howelletal+09,How11}, two
alternative models, the double
degenerate and sub-Chandra models for SNe Ia, have become more
scientifically interesting.  Theoretically, these both have
challenges.

The standard picture for double degenerate SN Ia (henceforth WDWD) has the
two white dwarfs spiraling toward each other as gravitational radiation
removes orbital energy from the system.  Here the white dwarfs can be more
moderate in mass, but the sum of their masses may exceed the
Chandrasekhar mass.  As the stars inspiral, the less massive star will
become tidally disrupted and the more massive star will accrete this
material.  A longstanding concern is that when this mass transfer
begins, thermonuclear burning can ignite at the edge of the star,
converting it to oxygen/neon/magnesium, and leading to the collapse of
the white dwarf into a neutron
star~\cite{saionomoto:2004,fryerdiehl:2008}, instead of an SNe Ia.
This system is inherently three-dimensional, and only through detailed
simulation can we understand the dynamics of the mass transfer, and
thereby assess the feasibility of this model.  Alternatives involving
(nearly) head-on collisions of white dwarfs may account for a few
events, but are unlikely to explain the majority.

The other alternative mechanism, sub-Chandra mass single degenerate SNe
Ia~\cite{fink:2010,shen:2010,sim:2012} (henceforth called the sub-Ch
model), has the advantage that systems with a moderate mass white dwarf
(0.8--1.0 solar masses) are known and abundant.  In addition, they are
able to reproduce what we know of delay-time distributions and SNe Ia
rates---something the Chandra model struggles with~\cite{ruiter:2011}.
The core idea of this model is that the white dwarf accretes a layer
of helium on its surface and a detonation ignites in this helium
layer.  This surface detonation can send a compression wave into the
underlying carbon/oxygen white dwarf, compressing the core to the point 
of igniting a detonation that unbinds the star. The main
problem with this ``double detonation'' 
model is that it is easier to ignite a detonation in
a more massive helium layer than a low mass helium layer, but too much
helium will lead to excessive production of iron-group elements when
compared to observations. Also, 
models have difficulty reproducing the intermediate mass elements 
(i.e.\ Si) seen in the spectra on standard SNe
Ia~\cite{hoeflich:1996,nugent:1997,kromer:2010}.  
Investigations into 
so-called ``.Ia'' explosions suggest the mass of the helium shell need
not be as large as previously assumed to trigger
runaway~\cite{bildsten:2007}.  Inspired by this, it has been suggested
that a detonation in a very thin He layer can trigger the detonation
of the core without over-producing surface iron-group
elements~\cite{fink:2010}, although detailed 1D stellar evolution
calculations suggest that these smallest mass He shells may ignite as
deflagrations or novae instead of as detonations, depending on the
properties of the system~\cite{woosleykasen:2010}.  If it can ignite
as a detonation, synthetic lightcurves can match
observations~\cite{kromer:2010}.  The challenge now becomes
understanding whether and under what conditions ignition can take
place in these thin shells when a realistic 3D convective field is
realized, as well as the subsequent evolution and character of that
ignition---this is what we will answer.

In past INCITE awards, we focused on the last hours of
convection, ignition, and the explosion in the Chandra model.  There
is an aspect of the Chandra model that we have not explored in
detail---this is the earliest stages of the convection in the white
dwarf, when neutrino losses in the reaction can alter the dynamics of
the convection (called the URCA process).  The only multi-dimensional
simulations of URCA~\cite{URCA} to date have been 2D, with only a
portion of the star modeled.  We can apply the same methodology we
used for the ignition calculations in the Chandra model during our
current INCITE award to study this problem in 3D.  This will be a
small focus of the proposed work.
