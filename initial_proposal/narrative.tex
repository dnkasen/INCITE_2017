\documentclass[11pt,letterpaper,english]{article}
\usepackage[T1]{fontenc} % Standard package for selecting font encodings
\usepackage{txfonts} % makes spacing between characters space correctly
\usepackage{xcolor} % Driver-independent color extensions for LaTeX and pdfLaTeX.
\usepackage{hyperref}  %The ability to create hyperlinks within the document

\usepackage{fancyhdr} % header footer placement

\usepackage[top=1in, bottom=1in, left=1in, right=1in] {geometry} % Margins
\usepackage{graphicx}   % Essential for adding images to you document.

\usepackage[numbers]{natbib}

%\usepackage{sectsty}
%% \sectionfont{\normalsize}
%% \subsectionfont{\normalsize}
%\subsubsectionfont{\normalsize \it}

\usepackage[small,compact]{titlesec}
\titlespacing{\section}{0pt}{0.5em}{0.25em}
\titlespacing{\subsection}{0pt}{0.5em}{0.125em}
\titlespacing{\subsubsection}{0pt}{0.5em}{0.25em}
\titlespacing{\paragraph}{0pt}{0.75em}{0.25em}

\setlength{\textfloatsep}{7pt}

\titleformat*{\subsubsection}{\itshape}

\let\oldthebibliography=\thebibliography
  \let\endoldthebibliography=\endthebibliography
  \renewenvironment{thebibliography}[1]{%
    \begin{oldthebibliography}{#1}%
      \setlength{\parskip}{0.1ex}% 
      \setlength{\itemsep}{0.0ex}% 
  }%
  {% 
    \end{oldthebibliography}% 
  }


% commas between multiple footnotes
\newcommand\fnsep{\textsuperscript{,}}

\usepackage{caption}
\captionsetup{labelsep=period}

\setlength{\parskip}{0.1\baselineskip}%
%\setlength{\parindent}{0pt}%

\input newcommands

%\raggedright

\begin{document}

\pagestyle{fancy} 
\lhead{Approaching Exascale Models of Astrophysical Explosions} 
\rhead{PI: Zingale} \renewcommand{%
\headrulewidth}{0.0pt}

\begin{center}
{\bf PROJECT NARRATIVE}
\end{center}


%-----------------------------------------------------------------------------
\section{SIGNIFICANCE OF RESEARCH}

%% Explain what advances you expect to be enabled by an INCITE award that
%% justifies an allocation of petascale resources (e.g., anticipated
%% impact on community paradigms, valuable insights into or solving a
%% long-standing challenge, etc). Place the proposed research in the
%% context of competing work in your discipline or business. The
%% information should be sufficient for peer review in your area of
%% research and also appropriate for general scientific review comparing
%% your proposal with proposals in other disciplines. Potential impact is
%% the predominant determinant for awards. This factor will be assessed
%% by a peer-review panel. Also list any previous INCITE award(s)
%% received and discuss the relationship to the work proposed. {\bf This
%%   section is typically about 4 pages.}

We propose a collaborative investigation of multiple types of stellar
explosions and their precursors using a suite of state-of-the-art
hydrodynamics codes.  Several progenitor models of Type Ia supernovae
will be explored, including their pre-explosion and explosive phases,
using our codes \maestro, \castro, and \flash.  X-ray bursts will also
be studied in their pre-explosive and explosive phases, using
\maestro\ and \castro.  The curious radiation-dominated systems like
the black widow pulsar will be explored with \castro, and
core-collapse supernovae will be studied using \chimera.  All
calculations will be three-dimensional and face the challenges of
capturing the effects of turbulence, instabilities, strong
gravitational interactions, nuclear reactions, and radiation.  These
challenges make these problems INCITE-class, and only the resouces
titan can provide will enable use to further our understanding.
Despite the broad suite of codes, there are common links, namely in
the microphysics (reactions, equations of state) and in our shared
approach to utilitizing the GPUs on titan.  This collaboration is an
expansion of our existing INCITE proposal, but we believe that it
represents the most efficient use of resources, allowing us to work
together as a team to create portable solvers to effectively make
use of this architecture and its successor.  These GPU-enabled
solvers will be made freely available.

Our publication record demonstrates that we have made productive use
of our current INCITE award, and we expect similar productivity
carrying forward.  Furthermore, this INCITE time will be used to train
the next generation of computational scientists---graduate students
feature prominently in the proposed work plan.


\subsection{Type Ia supernovae}

\input sneIa.tex


\subsection{Type I X-ray bursts}

\input xrb.tex


\subsection{Core-collapse supernovae}

\input ccsne.tex


\subsection{Black-widow pulsars}

\input bwp.tex


\subsection{\maestro\ and \castro}

The XRB, BWP, and most of the SNe Ia work will be performed with our
state-of-the-art simulation codes \maestro~\cite{multilevel} and
\castro~\cite{castro:I}, developed over the past 10 years in
collaboration with Lawrence Berkeley Lab.  \maestro\ is tuned to
efficiently model the highly subsonic convective flows that often
precede stellar explosions.  It accomplishes this by reformulating the
equations of hydrodynamics, filtering the soundwaves from the
equations of hydrodynamics, while keeping the compressibility effects
due to stratification and local heat release.  This allows it to take
much larger timesteps than traditional compressible codes.
\maestro\ has been used successfully to model the convection leading
to ignition in the Chandra model for SNe Ia, and will continue to be
used for the XRB, sub-Ch, and Urca convection simulations proposed
here.  \castro\ solves the fully compressible equations of
hydrodynamics, allowing it to model shocks and explosive phenomena.
It will be the simulation code for the white dwarf mergers and flame
spreading on XRBs.  Considerable development has been done during the
current allocation period to allow \castro\ to accurate model gravity
and rotation for a pair of WDs~\cite{katz:2016}.  In later years, \castro\
will see a role in carrying our sub-Ch convection calculations into
the explosion phase.  \maestro\ and \castro\ share the underlying
microphysics e.g., equation of state (EOS) and nuclear reaction
networks, as well as the underlying \boxlib\ library that manages
their adaptive mesh refinement (AMR) grid hierarchy.  This makes it
straightforward to transition a problem from \maestro\ to \castro, as
was done during our previous INCITE allocation when studying flame
propagation in Chandra model SNe Ia~\cite{Mal14}.  Both codes are
already up and running on our target platform, titan (OLCF).
Additionally, both codes are publicly available%
\footnote{https://github.com/BoxLib-Codes}---any performance or
physics improvements developed under this INCITE award will become
part of the public releases, benefiting the community at large.

\subsection{\flash}

\subsection{\chimera}


\subsection{Work under previous INCITE awards}

We are currently in the last year of a 2-year INCITE award (AST106;
PI: Zingale; allocation on OLCF/titan) that focused on XRBs and SNe
Ia.  Many of the current investigators have also collaborated on other
previous INCITE awards.  \MarginPar{number of papers}

{\bf \maestro\ convection models}:
%

Adam's paper + results

{\bf \castro\ WD merger models}:
%

Max's paper + results



{\bf \maestro XRB modules}:

We have completed the first-ever three-dimensional calculation of
convection in the accreted layer on a neutron star as a model for the
pre-explosion phase of an XRB.  This calculation modeled a small
portion of the star (30~m on a side) with a mixed H/He layer,
resolving the burning scales (6 cm resolution) and evolving 11
isotopes describing hot-CNO burning and some branches to the
rp-process.  We demonstrated that the convective layer becomes
turbulent with a cascade that matches Kolmogorov isotropic turbulence.



{\bf GPU work}:

We continued to make substantial progress on offloading the
microphysics in our codes to the GPUs using OpenACC.  Our focus is to
put the nuclear reaction networks onto the GPUs, since they are local,
compute-heavy, standalone physics solvers that take an appreciable
amount of our runtime. Our strategy is to put the entire integration,
the ODE timestepping driver and evaluation of the righthand-side and
Jacobian onto the GPUs.  Previously we demonstrated a speed-up when
using a simple network and a first-order backward-difference
integration scheme.  This past year, we have worked closely with OLCF
staff, including participating in a GPU hackathon, and have our
high-order integrators working on the GPUs.  Our collaboration
with OLCF continues, Co-I Jacobs will be traveling to OLCF next week
to work on profiling on the GPUs to understand branching issues and
further the optimization.




\subsection{Significance of our proposed work}

To advance the state of the knowledge of SNe Ia and XRBs, we will
carry out the following sets of simulations (all in 3D):
\begin{tightitem}
\item \castro\ simulations of white dwarf mergers
\item Full star \maestro\ sub-Ch models with 
  detailed nucleosynthesis and realistic initial models.
\item The largest-domain-to-date 3D resolved \maestro\ XRB convection
  calculations.
\item \castro\ simulations of burning fronts on neutron stars
  as a model for XRB flame propagation
\item \maestro\ simulations of convective Urca in white dwarfs
\end{tightitem}

We are the only group in the world modeling the detailed convection
with realistic nuclear physics in XRBs and the sub-Ch SNe Ia.
In all simulations we will push the size of the nuclear reaction networks
to accurately capture the nucleosynthesis (as discussed later, GPUs will
be used for this task).

XRBs are important probes of the nuclear equation of \MarginPar{needs updating}
state and the nucleosynthesis that
takes place will involve the nuclei that are a target of the DOE FRIB
experiment.  Our simulations will advance our understanding on the
burning dynamics and tell us (1) whether any burning products can be
carried to the photosphere, altering the interpretation of
observations; (2) how the full 3D treatment of convection modifies the
nucleosythesis, allowing us to provide feedback to the 1D modelers;
(3) how to create a sub-grid model to enable larger scale simulations;
and (4) ultimately, in year 3 when we are able to implement a sub-grid
model, start to learn how turbulence in the burning alters the
spreading of the front across the neutron star.  With \maestro, we are
in the unique position to address each of these points.

While several groups are investigating explosions in the sub-Ch model,
we are the only group that is modeling the 3D convective field and
ignition that precedes the explosion.  Much like our previous INCITE
work in carrying \maestro\ convection calculations of the Chandra
model into the explosion phase with \castro, our similar work here
(proposed for year 3) for the sub-Ch model will be unmatched.
Initially we will focus on using more realistic models of WDs
generated from stellar evolution codes (we currently use simple
parametrized models), and we will answer the question of which
configurations (WD mass and He mass) lead to ignitions as well as the
timing and geometry of this ignition.  Along the way we will switch
from our simplified 4-isotope network to a more general in-situ
network to better understand the sensitivity of our results to the
nuclear physics, metallicity, and trace abundances of other species.
In addition, we will complete our implementation of nucleosynthetic
post-processing using Lagrangian tracer particles to allow offline
exploration.  This is key, as the exact composition of the helium
layer at detonation can have profound effects on the subsequent
observations, impacting the model's viability as a SNe Ia
progenitor~\cite{kromer:2010}.  All simulations will be 3D, and the
vast majority will model the entire star.  These results will provide
the foundation for \castro\ simulations capable of determining if the
ignition evolves as a detonation and that detonation's subsequent
evolution with realistic 3D initial conditions.
 
Several groups are pursuing WDWD \MarginPar{needs updating}
mergers~\cite{yoon:2007,motl:2007,loren-aguilar:2009,shenetal+11} or
collisions~\cite{raskinetal+10,loren-aguilar:2010,rosswog:2009}. The
majority of these have used smoothed particle hydrodynamics (SPH), a
gridless alternative to the methods we use here.  SPH is known to have
trouble capturing instabilities and has low resolution in regions of
low density---precisely the regions where the stars make contact.  Our
grid--based simulations will provide an important counterpart to
existing simulations.  With the power of AMR, we can push far beyond
the resolutions of the grid-based simulations in the current
literature to levels necessary to assess whether an explosion upon
contact is feasible. \castro\ is ready for these simulations---we have
made the necessary changes for isolated gravity boundary conditions,
and while we will start with simple WD models on the grid, we are
nearly complete in implementing a self-consistent method to initialize a binary
pair of WDs on our grid using the full stellar equation of state.



%-----------------------------------------------------------------------------
\section{RESEARCH OBJECTIVES AND MILESTONES }  

%% Describe the proposed research, including its goals, milestones and
%% the theoretical and computational methods it employs. Goals and
%% milestones should articulate simulation and developmental objectives
%% and be sufficiently detailed to assess the progress of the project for
%% each year of any allocation granted. Milestones should correlate with
%% those in the milestone table. It is especially important that you
%% provide clear connections between the project's overarching
%% milestones, the planned production simulations, and the compute time
%% expected to be required for these simulations (e.g., should correlate
%% with those in the ``Use of Resources Requested'' section below). {\bf
%%   This section is typically about 6 pages.}

\input objectives


%-----------------------------------------------------------------------------
\input readiness
\bibliographystyle{unsrtnat}
\bibliography{refs}




\end{document}
