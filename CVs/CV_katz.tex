\documentclass[11pt,letterpaper,english]{article}
\usepackage[T1]{fontenc} % Standard package for selecting font encodings
\usepackage{txfonts} % makes spacing between characters space correctly
\usepackage{xcolor} % Driver-independent color extensions for LaTeX and pdfLaTeX.
%\usepackage{blindtext} % To create text
%\usepackage{mdwlist} % mdwlist for compact enumeration/list items 
%\usepackage[pagestyles]{titlesec} % related with sections—namely titles, headers and contents
\usepackage{fancyhdr} % header footer placement

\usepackage[top=1in, bottom=1in, left=1in, right=1in] {geometry} % Margins
\usepackage{graphicx}   % Essential for adding images to you document.

\usepackage{sectsty}
\sectionfont{\large}
\subsectionfont{\normalsize}
\subsubsectionfont{\normalsize \it}

\usepackage{caption}
\usepackage{hyperref}
\captionsetup{labelsep=period}

\pagestyle{fancy} % allows you to use the header and footer commands 

\raggedright
\begin{document}

\setlength{\parindent}{0in} % Amount of indentation at the first line of a paragraph.

\pagestyle{fancy} \lhead{Approaching Exacale Models of Astrophysical Explosions} \rhead{Michael Zingale} \renewcommand{%
\headrulewidth}{0.0pt}



\centering {\bf Curriculum Vitae: Maximilian Katz }\\
Department of Physics and Astronomy, Stony Brook University\\
Stony Brook, NY 11794-3800\\
Phone: (845) 665-1393 \hspace{0.1in} E-mail: maximilian.katz@stonybrook.edu \smallskip

\begin{flushleft} {\bf Professional Preparation}
{\parindent 16pt

Ph.D. in Physics and Astronomy, Stony Brook University, in preparation (expected 2016)\\ 
M.S. in Physics, Rensselaer Polytechnic Institute (2011) \\ 
B.S. in Physics, Rensselaer Polytechnic Institute (2010) \\ 
}

\vspace{.04in}
{\bf Appointments}
{\parindent 16pt

2013--present: \textit{Graduate Research Assistant}, Dept. of Physics and Astronomy, Stony Brook University \\ 
2011--2012: \textit{Teaching and Learning Assistant}, Advising and Learning Assistance Center, RPI \\
2007--2011: \textit{Research Assistant}, Dept. of Physics, Applied Physics and Astronomy, RPI \\
}

\vspace{.04in}
{\bf Five Publications Most Relevant to This Proposal}
\vspace{-6pt}
\begin{enumerate} \itemsep1pt \parskip0pt \parsep0pt
\item {\it Double White Dwarf Mergers with CASTRO}, Katz, M.~P., Zingale, M., Calder, A., \& Swesty, F.~D.\ 2013, American Astronomical Society Meeting Abstracts, 221, \#253.21 \\ 
\item {\it Driven Hydromagnetic Waves and Shocks in Dusty Interstellar Clouds} Katz, M.~P., Ciolek, G.~E., \& Roberge, W.~G.\ 2011, Bulletin of the American Astronomical Society, 43, \#251.03 \\ 
\item {\it Diffusion, Self-Similarity, and the Formation of Multifluid Shock Waves} Ciolek, G.~E., Roberge, W.~G., \& Katz, M.~P.\ 2014, American Astronomical Society Meeting Abstracts \#223, 223, \#454.26 \\
%\item Publication most relevant to this proposal \\ 
%\item Publication most relevant to this proposal \\ 
\end{enumerate} 

\vspace{-6pt}
{\bf Research Interests and Expertise}
{\parindent 16pt

I am interested in applying computational science skills to problems that demand interdisciplinary scientific skillsand knowledge. My current focus is on stellar astrophysics. My Ph.D. dissertation is on the subject of determining the progenitor systems of the astrophysical explosions known as Type Ia supernovae. To this end, I contribute to the development of the code I use for the research, and I am working on a number of algorithms to improve the study of systems that may result in these explosions. In particular, I have worked on upgrades to the gravity, thermodynamics, and hydrodynamics modules of our code. I also have spent time working on one-dimensional stellar evolution modelling. My undergraduate research focused on magnetohydrodynamics in the interstellar medium.
}

%Synergistic Activities. A list of up to five examples that demonstrate
%the broader impact of the individual's professional and scholarly
%activities that focuses on the integration and transfer of knowledge
%as well as its creation.

\vspace{.04in}
{\bf Synergistic Activities}
\vspace{-6pt}
\begin{enumerate} \itemsep1pt \parskip0pt \parsep0pt
\item Contributor to the publicly available hydrodynamics code \href{https://ccse.lbl.gov/Downloads/downloadCASTRO.html}{CASTRO}. \\ 
\item Contributor to the publicly available stellar evolution code \href{http://mesa.sourceforge.net/}{MESA}. Participated in the \href{http://cococubed.asu.edu/mesa_school_website/mesa_summer_school/Home.html}{2012 MESA Summer School} and was invited to be a Teaching Assistant in the \href{http://cococubed.asu.edu/mesa_summer_school_2013/Home.html}{2013 MESA Summer School}. \\ 
\item Participated in the 2013 \href{http://extremecomputingtraining.anl.gov/}{Argonne Training Program on Extreme Scaling Computing} and the 2014 \href{http://summerschool.niif.hu/}{International HPC Summer School}. \\
\item Instructor for the Astrophysics course at Johns Hopkins University's \href{http://cty.jhu.edu/}{Center for Talented Youth} (2011--2013); this course covered college-level astrophysics for bright secondary school students. \\ 
%\item Text \\ 
%\item Text \\ 
\end{enumerate} 

\vspace{-6pt}
{\bf Collaborators ({\emph{past 5 years including name and current institution}})} 
{\parindent 16pt

Ann Almgren, Lawrence Berkeley National Laboratory\\ 
Alan Calder, Stony Brook University\\
Glenn Ciolek, Rensselaer Polytechnic Institute\\
Bill Paxton, University of California, Santa Barbara\\
Wayne Roberge, Rensselaer Polytechnic Institute\\
Douglas Swesty, Stony Brook University\\
Michael Zingale, Stony Brook University\\
}


\end{flushleft}

\end{document}
