\documentclass[11pt,letterpaper,english]{article}
\usepackage[T1]{fontenc} % Standard package for selecting font encodings
\usepackage{txfonts} % makes spacing between characters space correctly
\usepackage{xcolor} % Driver-independent color extensions for LaTeX and pdfLaTeX.
%\usepackage{blindtext} % To create text
%\usepackage{mdwlist} % mdwlist for compact enumeration/list items 
%\usepackage[pagestyles]{titlesec} % related with sections—namely titles, headers and contents
\usepackage{fancyhdr} % header footer placement

\usepackage[top=1in, bottom=1in, left=1in, right=1in] {geometry} % Margins
\usepackage{graphicx}   % Essential for adding images to you document.

\usepackage{sectsty}
\sectionfont{\large}
\subsectionfont{\normalsize}
\subsubsectionfont{\normalsize \it}

\usepackage{caption}
\captionsetup{labelsep=period}

\usepackage{hyperref}

\pagestyle{fancy} % allows you to use the header and footer commands 

\raggedright
\begin{document}

\setlength{\parindent}{0in} % Amount of indentation at the first line of a paragraph.

\pagestyle{fancy} \lhead{Approaching Exascale Models of Astrophysical Explosions} \rhead{M.~Zingale} \renewcommand{%
\headrulewidth}{0.0pt}



\centering {\bf Curriculum Vitae: Christopher Malone}\\
{ CCS-2, MS: P365, Los Alamos National Laboratory, Los Alamos, NM 87545} \\
{\it phone:}~(727) 385-2501 \hskip 2mm
{\it e-mail:}~cmalone@lanl.gov \hskip 2mm 

\begin{flushleft} {\bf Professional Preparation}
{\parindent 16pt

Ph.D. in Physics and Astronomy, Stony Brook University (2011)\\ 
M.A. in Physics, Stony Book University (2007)\\ 
B.S. in Physics, Florida Institute of Technology (2005)\\ 
B.S. in Astronomy and Astrophysics (2005)\\
}

\vspace{.04in}
{\bf Appointments}
{\parindent 16pt

2014--present: {\em Metropolis Postdoctoral Fellow}, Los Alamos National Laboratory \\ 
2011--2014: {\em Postdoctoral Researcher}, Dept. of Astronomy and Astrophysics, UC Santa Cruz\\ 
2006--2011: {\em Graduate Student Researcher}, Dept. of Physics and Astronomy, Stony Brook University  \\ 
2005--2005: {\em Research Scientist}, Florida Space Research Institute \\
2003--2005: {\em Research Assistant}, Dept. of Physics and Space Sciences, Florida Institute of Technology
}

\vspace{.04in}
{\bf Five Publications Most Relevant to This Proposal}
\vspace{-6pt}
\begin{enumerate} \itemsep1pt \parskip0pt \parsep0pt
\item {\it Multidimensional Modeling of Type I X-ray
  Bursts. II. Two-Dimensional Convection in a Mixed H/He Accretor, }
  C.~M.~Malone, M.~Zingale, A.~Nonaka, A.~S.~Almgren, \& J.~B.~Bell,
  2014, ApJ, 788, 115.

\item {\it The Deflagration Stage of Chandrasekhar Mass Models For
  Type Ia Supernovae: I. Early Evolution, } C.~M.~Malone, A.~Nonaka,
  S.~E.~Woosley, A.~S.~Almgren, J.~B.~Bell, S.~Dong, \& M.~Zingale,
  2014, ApJ, 782, 11.

\item {\it Low Mach Number Modeling of Convection in Helium Shells on
  Sub-Chandrasekhar White Dwarfs. I. Methodology,} M.~Zingale,
  A.~Nonaka, A.~S.~Almgren, J.~B.~Bell, C.~M.~Malone, \&
  R.~J.~Orvedahl, 2013, ApJ, 764, 97.

\item {\it Multidimensional Modeling of Type I X-ray
  Bursts. I. Two-Dimensional Convection Prior to the Outburst of a
  Pure $^4$He Accretor, } C.~M.~Malone, A.~Nonaka, A.~S.~Almgren,
  J.~B.~Bell, \& M.~Zingale, 2011, ApJ, 728, 118.

\item {\it MAESTRO: An Adaptive Low Mach Number Hydrodynamics
  Algorithm for Stellar Flows,} A.~Nonaka, A.~S.~Almgren, J.~B.~Bell,
  M.~J.~Lijewski, C.~Malone, \& M.~Zingale, 2010, ApJS, 188, 358.

\end{enumerate} 

\vspace{-6pt}
{\bf Research Interests and Expertise}
{\parindent 16pt

My interests involve computational simulation of explosive phenomena.
For the most part, this has involved thermonuclear astrophysical
applications.  In particular, I have worked on applying the low Mach
number hydrodynamics code Maestro to convection problems in both Type
I X-ray bursts and Type Ia supernovae, as well as working with
compressible flame models in Type Ia supernovae.  More recently, I've
begun an interest in modeling terrestrial high explosives.

}

%Synergistic Activities. A list of up to five examples that demonstrate
%the broader impact of the individual's professional and scholarly
%activities that focuses on the integration and transfer of knowledge
%as well as its creation.

\vspace{.04in}
{\bf Synergistic Activities}
\vspace{-6pt}
\begin{enumerate} \itemsep1pt \parskip0pt \parsep0pt

\item Contributor to the publicly available Maestro hydrodynamics
  code, \url{http://bender.astro.sunysb.edu/Maestro/}\\

\item Contributor to the publicly available Castro hydrodynamics code,
  \url{https://ccse.lbl.gov/Downloads/downloadCASTRO.html} \\

\item Contributor to the publicly available yt visualization code,
  \url{http://yt-project.org/} \\

\item External reviewer for NASA Earth and Space Science Fellowship. \\ 
\item Referee for ApJL. \\ 
\end{enumerate} 

\vspace{-6pt}
{\bf Collaborators ({\emph{past 5 years including name and current institution}})} 
{\parindent 16pt


Ann Almgren (LBL), 
John Bell (LBL),
Shawfeng Dong (UCSC),
Michael Lijewski (LBL),
Haitao Ma (unknown),
Andy Nonaka (LBL), 
Ryan Orvedahl UC Boulder),
Stan Woosley (UCSC),
Mike Zingale (Stony Brook)

past grad advisor: Mike Zingale (Stony Brook); past postdoc advisor: Stan Woosley (UCSC); current postdoc advisors: Chris Fryer \& Aimee Hungerford (LANL).
}


\end{flushleft}

\end{document}
