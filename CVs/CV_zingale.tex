\documentclass[11pt,letterpaper,english]{article}
\usepackage[T1]{fontenc} % Standard package for selecting font encodings
\usepackage{txfonts} % makes spacing between characters space correctly
\usepackage{xcolor} % Driver-independent color extensions for LaTeX and pdfLaTeX.
%\usepackage{blindtext} % To create text
%\usepackage{mdwlist} % mdwlist for compact enumeration/list items 
%\usepackage[pagestyles]{titlesec} % related with sections—namely titles, headers and contents
\usepackage{fancyhdr} % header footer placement

\usepackage[top=1in, bottom=1in, left=1in, right=1in] {geometry} % Margins
\usepackage{graphicx}   % Essential for adding images to you document.

\usepackage{sectsty}
\sectionfont{\large}
\subsectionfont{\normalsize}
\subsubsectionfont{\normalsize \it}

\usepackage{caption}
\captionsetup{labelsep=period}

\usepackage{hyperref}

\pagestyle{fancy} % allows you to use the header and footer commands 

\raggedright
\begin{document}

\setlength{\parindent}{0in} % Amount of indentation at the first line of a paragraph.

\pagestyle{fancy} \lhead{Approaching Exascale Models of Astrophysical Explosions
} \rhead{M.~Zingale} \renewcommand{%
\headrulewidth}{0.0pt}


%\thispagestyle{plain}

\centering {\bf Curriculum Vitae: Michael Zingale}\\
{Department of Physics and Astronomy, Stony Brook University, Stony
  Brook, NY 11794-3800} \smallskip
{{\it phone:}~(631) 632-8225 \hskip 2mm
{\it e-mail:}~Michael.Zingale@stonybrook.edu \hskip 2mm \\[-0.25em]
{\it web:}~\url{http://www.astro.sunysb.edu/mzingale/}}

\begin{flushleft} {\bf Professional Preparation}
{\parindent 16pt

PhD in Astronomy and Astrophysics, University of Chicago (2000)\\ 
MS in Astronomy and Astrophysics, University of Chicago (1998)\\ 
BS in Physics and Astronomy, University of Rochester (1996)\\ 
}

\vspace{.04in}
{\bf Appointments}
{\parindent 16pt

2012--present: {\em Associate Professor of Physics and Astronomy}, Stony Brook University \\ 
2014--present: {\em Affiliate}, Institute for Advanced Computational Science, Stony Brook University \\ 
2006--2011:  {\em Assistant Professor of Physics and Astronomy}, Stony Brook University \\ 
2001--2005: {\em Postdoctoral Researcher}, SciDAC Supernova Science Center, UC Santa Cruz \\
2000--2001: {\em Research Associate}, Flash Center, University of Chicago \\
1997--2000: {\em Graduate Student Researcher}, Flash Center, University of Chicago 
}

\vspace{.04in}
{\bf Five Publications Most Relevant to This Proposal}
\vspace{-6pt}
\begin{enumerate} \itemsep1pt \parskip0pt \parsep0pt
\item {\it Multidimensional Modeling of Type I X-ray
  Bursts. II. Two-Dimensional Convection in a Mixed H/He Accretor, }
  C.~M.~Malone, M.~Zingale, A.~Nonaka, A.~S.~Almgren, \& J.~B.~Bell,
  2014, ApJ, 788, 115.

\item {\it The Deflagration Stage of Chandrasekhar Mass Models For
  Type Ia Supernovae: I. Early Evolution, } C.~M.~Malone, A.~Nonaka,
  S.~E.~Woosley, A.~S.~Almgren, J.~B.~Bell, S.~Dong, \& M.~Zingale,
  2014, ApJ, 782, 11.

\item {\it Low Mach Number Modeling of Convection in Helium Shells on
  Sub-Chandrasekhar White Dwarfs. I. Methodology,} M.~Zingale,
  A.~Nonaka, A.~S.~Almgren, J.~B.~Bell, C.~M.~Malone, \&
  R.~J.~Orvedahl, 2013, ApJ, 764, 97.

\item {\it CASTRO: A New Compressible Astrophysical
  Solver. I. Hydrodynamics and Self-Gravity,} A.~S.~Almgren,
  V.~E.~Beckner, J.~B.~Bell, M.~S.~Day, L.~H.~Howell, C.~C.~Joggerst,
  M.~J.~Lijewski, A.~Nonaka, M.~Singer, \& M.~Zingale, 2010, ApJ, 715,
  1221.

\item {\it MAESTRO: An Adaptive Low Mach Number Hydrodynamics
  Algorithm for Stellar Flows,} A.~Nonaka, A.~S.~Almgren, J.~B.~Bell,
  M.~J.~Lijewski, C.~Malone, \& M.~Zingale, 2010, ApJS, 188, 358.

\end{enumerate} 

\vspace{-6pt}
{\bf Research Interests and Expertise}
{\parindent 16pt

I am interested in developing and
  applying computational hydrodynamics algorithms to problems in
  nuclear astrophysics.  A large part of this work is the development
  of low Mach number hydrodynamics algorithms suited toward long-time
  evolution in astrophysical flows.  The low Mach number simulation
  code Maestro (developed together with collaborators at LBNL) has
  been applied to a variety of problems to model convection in stellar
  environments, including Type Ia supernovae, X-ray bursts, novae,
  and massive star evolution.  Maestro is publicly available.

}

\vspace{.04in}
{\bf Synergistic Activities}
\vspace{-6pt}
\begin{enumerate} \itemsep1pt \parskip0pt \parsep0pt
\item Recently taught the graduate courses: {\em Numerical Methods for
  (Astro)Physics} and {\em Python for Scientific Computing}. \\

\item Co-developer of the publicly-available low Mach number
  hydrodynamics code Maestro,
  \url{http://bender.astro.sunysb.edu/Maestro/} \\

\item Developed a freely-available python-based teaching code showing
  the basic methods in hydrodynamics (advection, compressible flow, 
  diffusion, multigrid, and incompressible flow), along with a detailed
  set of lecture notes: \url{http://bender.astro.sunysb.edu/hydro_by_example/} \\ 
\item Annual public lectures as part of the Stony Brook Astronomy Open Night program
 \\
 
\item Referee for ApJ, A\&A, Nature, MNRAS, JCP; reviewer for NASA (ATP), NSF (PRAC, OCI, AAP), DOE (NP) \\ 
\end{enumerate} 

\vspace{-6pt}
{\bf Collaborators ({\emph{past 5 years including name and current institution}})} 
{\parindent 16pt

Ann Almgren (LBL),
Andy Aspden (Cranfield),
Vince Beckner (LBL),
John Bell (LBL),
Adam Burrows (Princeton),
Alan Calder (Stony Brook),
Andrew Cumming (McGill),
Marc Day (LBL),
Gary Glatzmaier (UCSC),
Alex Heger (Monash),
Rob Hoffman (LLNL),
Louis Howell (LLNL),
Dan Kasen (Berkeley),
Jim Lattimer (Stony Brook),
Mike Lijewski (LBL),
Haitao Ma (unknown),
Chris Malone (LANL),
Eric Myra (Michigan),
Andy Nonaka (LBL),
Peter Nugent (LBL),
Fritz Roepke (MPA Garching),
Doug Swesty (Stony Brook),
Stan Woosley (UCSC)

past grad advisor: Jim Truran (Chicago); past postdoc advisor: Stan Woosley (UCSC).

}


\end{flushleft}

\end{document}
